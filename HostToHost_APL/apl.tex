\documentclass[11pt, a4paper]{article}

\input{"/home/zamza/Documents/HS/Notes Template/preamble.tex"}

\newcommand{\code}[1]{#1}

\begin{document}

\begin{center}
  \Large{Host-to-Host-Kommunikation im Rahmen der Netzwerkprogrammierung und Untersuchung der maximalen TCP-Segmentgröße}
\end{center}

\begin{flushright}
  R. Grünert\\
  S. Klobe\\
  \today
\end{flushright}

Zur erfolgreichen Kommunikation zwischen zwei Hosts mit z. B. \code{echotcp} bzw. \code{daytimetcp} bedarf es mehrerer Schritte
zur Konfiguration.

\section{IPv4 Adresszuweisung}

Da im Modul ENP die Netzwerkkommunikation auf
Schicht 3 über IPv4 gelehrt wird, benötigt zumindest der Server-Host
eine solche öffentliche Adresse. Da man heutzutage bei den meisten
Providern mit \emph{Dual Stack Lite} (DS-Lite) angebunden wird und
daher nur eine IPv6-Adresse erhält, ist in diesem Fall eine Anfrage beim Provider nötig, um eine IPv4 Adresse zugewiesen zu bekommen.
Im Falle unseres Providers (Vodafone) erfolgte dies durch einen einfachen Anruf.\\

Gelingt keine IPv4-Zuweisung, bleiben nur Ausweichmöglichkeiten, wie z. B. VPN-Verbindung beider Netzwerke.\\

Mithilfe einer Website zur Ermittlung der eigenen öffentlichen IP-Adresse kann im vorher-nachher-Vergleich die Zuweisung überprüft werden \footnote{z. B. mit https://www.whatismyip.com/}.


\section{Port-Forwarding}
Aus Sicht des inside-global Netzwerkes gibt es noch keine Möglichkeit,
um den Server von anderen Hosts im Heimnetzwerk zu unterscheiden.
Der Router sollte daher so konfiguriert werden, dass er den Traffic,
der an den Port des Servers gerichtet ist auch an den Server
weiterleitet (Port-Forwarding).

\subsection{Ermittlung der Router-IP-Adresse}
Über das Webinterface des Routers kann man das Port-Forwarding einstellen. Um die lokale IP-Adresse des
Routers zum Erreichen des Webinterfaces herauszufinden, gibt es betriebssystemabhängig verschiedene Möglichkeiten.

\paragraph{Router-IP unter Windows}
Über den CMD-Befehl \code{ipconfig} findet man unter dem Eintrag
\glqq Default Gateway\grqq bzw \glqq Standardgateway\grqq die gewünschte Routeradresse (Abb. \ref{gwin}).

\begin{figure}[H]
\includegraphics[width=\textwidth]{graphics/gatewayip_windows}
\caption{CMD-Ausgabe des ipconfig-Befehls unter Windows}\label{gwin}
\end{figure}

\paragraph{Router-IP unter Linux}
Unter Linux kann man den Befehl \code{ip r} oder \code{ip route} nutzen, um die Routeradresse herauszufinden.

\begin{figure}[H]
\centering
\includegraphics[width=0.618\textwidth]{graphics/gatewayip_linux}
\caption{Ausgabe des ipconfig-Befehls unter Linux}\label{glin}
\end{figure}



\subsection{Aufruf des Webinterfaces}
Über die oben ermittelte IP-Adresse kann man im Browser das Webinterface des Routers aufrufen. Das Passwort findet man i.d.R. auf dem Router selbst.

\subsection{Einstellung des Port-Forwardings}
Die genaue Einstellung des Port-Forwardings unterscheidet sich natürlich je nach Gerät, im allgemeinen müssen dann
aber nur ein Port(bereich) und eine gewünschte Ziel-IP-Adresse
angegeben werden. Ein Beispiel ist in Abb. \ref{forv} zu sehen.\\

Die Ziel-IP-Adresse ist die Adresse des Hosts auf dem das Serverprogramm läuft. Man findet sie unter Windows ebenfalls über den \code{ipconfig}-Befehl und unter Linux z.B. über \code{ip a}. Da die Adresszuweisung meist standardmäßig über DHCP stattfindet, sollte darauf geachtet werden, dass sich die beim Port-Forwarding angegebene Adresse ändern und daher eine Neukonfiguration notwendig sein kann.

\begin{figure}[H]
\centering
\includegraphics[width=\textwidth]{graphics/forv}
\caption{Beispiel für Port-Forwarding Ansicht eines Vodafone-Routers}\label{forv}
\end{figure}

\section{Firewall}

\begin{figure}[H]
\centering
\includegraphics[width=\textwidth]{graphics/firewall_aus}
\caption{}\label{fwaus}
\end{figure}

\section{Virtualbox}
Da TeilnehmerInnen des Moduls ENP meist eine virtuelle Maschine
(Oracle Virtualbox) nutzen, um unter Linux zu arbeiten, benötigt man
diesem Fall zusätzliche Konfiguration unter Virtualbox.

\subsection{Port-Forwarding}
Möglicherweise bedarf es einem Neustart von Windows.

\section{}
\begin{itemize}
  \item firewall
  \item virtualbox settings (Netzwerk -> Port Forwarding) ( Neustart? )
  \item richtige adresse im serverprogramm einstellen (Ausgabe!)
  \item port forwarding am router
  \item screenshots wie es aussieht wenn was schiefläuft (firewall, nachricht wird abgelehnt, etc)
\end{itemize}

\end{document}
